\documentclass[standalone]{beamer}

\begin{document}
\section{轉移點單調優化}

\begin{frame}{\btitle{例題}}
  \begin{problem}[ZJ a146 Sliding Window]
    令 $F(i, j)$ 為一個對所有 $1 \leq i \leq j \leq N$ 的正整數 $i, j$ 都有定義的函數。對於所有的 $1 \leq j \leq N$,求 $\min_{i \leq j}F(i, j)$ 或 $\max_{i \leq j}F(i, j)$。
  \end{problem}
  這樣看起來有點抽象,我們來實際看一個例子:
\end{frame}

\begin{frame}{\btitle{例題}}
  \begin{problem}[ZJ a146 Sliding Window]
    有 $N$ 個位於一維數線上的餐廳,由左至右編為 $1$ 到 $N$ 號,其中 $i$ 號餐廳與 $i - 1$ 號餐廳的距離為 $A_i$。你有 $M$ 張餐卷,第 $i$ 張可以兌換第 $i$ 種食物且只能使用一次。已知這 $N$ 家餐廳都有販售這 $M$ 種食物,且第 $i$ 家餐廳的第 $j$ 種食物的好吃度為 $B_{i, j}$。你可以從任意間餐廳開始,請問「吃到食物的好吃度總和 $-$ 總行走距離」的最大值為何。
    
    \begin{itemize}
      \item $1 \leq N \leq 5000$
      \item $1 \leq M \leq 200$
    \end{itemize}
  \end{problem}
\end{frame}

\begin{frame}{\btitle{例題}}
  \begin{itemize}
    \item 首先可以知道,當決定了要在哪些餐廳買食物之後,最佳的策略一定是從最左的直接走到最右邊的餐廳,不會來回行走
    \item 因此,定義 $F(i, j)$ 為從 $i$ 號餐廳開始,結束在 $j$ 號餐廳的最大價值。則 $F(i, j)$ 可以被寫成
    \item $F(i, j) = \left(\sum_{f = 1}^{M}\max_{i \leq k \leq j}B_{k, f}\right) - \left(\sum_{k = i + 1}^{j}A_k\right)$
    \item 其中 $i$ 與 $j$ 之間各種食物的最大好吃度可以透過 RMQ 預處理,$i$ 與 $j$ 的距離也可以透過前綴和達到 $O(1)$,因此可以在 $O(M)$ 的時間內計算好 $F(i, j)$
  \end{itemize}
\end{frame}

\begin{frame}{\btitle{例題}}
  那該如何有效率的計算 $\max_{i \leq j}F(i, j)$,或是更一般的,對所有餐廳 $j$ 求出以它當終點的最大價值呢 $V_j$?這裡我們必須用到一個性質:
  \begin{theorem}[性質]
    令 $H_j$ 為以 $j$ 當終點情況下的最佳起點,亦即:

    \[ H_j = \mathop{\arg\max}_{i \leq j}F(i, j) \]

    則 $H_j \leq H_{j + 1}$。也就是說當終點向右時,最佳的起點只會往右或不動
  \end{theorem}
\end{frame}

\begin{frame}{\btitle{演算法}}
\begin{itemize}
  \item 有了這個性質之後,可以考慮下列的分治算法:
  \item 假設現在我們想要計算所有 $L \leq j \leq R$ 的 $V_j$,並且我們知道所有在範圍內的 $H_j$ 都滿足 $L^\prime \leq H_j \leq R^\prime$ 
  \item 令 $M = \lfloor\frac{L + R}{2}\rfloor$,那我們在 $O(R^\prime - L^\prime)$ 次計算 $F$ 值之內,枚舉 $[L^\prime, \min(R^\prime, M)]$ 之間的 $i$ 並找出最大的 $F(i, j)$ 以及對應的 $H_M$
  \item 根據單調性,我們知道 $[L, M - 1]$ 的最佳起點會落在 $[L^\prime, H_M]$ 、 $[M + 1, R]$ 的最佳起點會落在 $[H_M, R^\prime]$。因此我們分別遞迴計算子問題 $\{L, M - 1, L^\prime, H_M\}$ 及 $\{M + 1, R, H_M, R^\prime\}$
  \item 複雜度為 $O(N \log N \times F())$,其中 $F$ 是算一個 $F(i, j)$ 的值的複雜度
\end{itemize}
\end{frame}

\begin{frame}[fragile]{\btitle{演算法}}
  \begin{minted}[breaklines]{cpp}
    void dc(int L, int R, int best_L, int best_R) {
      if (L > R) return;
      int mid = (L + R) >> 1;
      for (int i = best_L; i <= best_R; ++i) {
        if (h[mid] == 0 || f(mid, i) > f(mid, h[mid])) h[mid] = i;
      }
      dc(L, mid - 1, best_L, h[mid]);
      dc(mid + 1, R, h[mid], best_R);
    }
  \end{minted}
\end{frame}

\begin{frame}{\btitle{CDQ 配轉移點單調}}
  \begin{itemize}
    \item 有了轉移點單調這個好東西後,我們可以回頭看一次 1D / 1D 凸單調
    \item 應該不難發現,\textbf{1D / 1D 凸單調也是具有轉移點單調}這個性質的,但是在有些時候,\textbf{轉移來源必須是在線的},導致我們無法快樂直接套用轉移點單調
    \item 但其實,這個世界並沒有這麼不美好。這個問題可以用類似斜率優化 CDQ 的思維,來多花一點時間解決!假設我們現在正在算 $[L, R]$ 的答案,算法過程如下:
  \end{itemize}
\end{frame}

\begin{frame}{\btitle{CDQ 配轉移點單調}}
  \begin{itemize}
    \item 如果 L == R,結束遞迴
    \item 否則,依序執行以下步驟:
    \begin{itemize}
      \item 遞迴計算左半部 $[L, mid]$ 的 DP 值。
      \item 用左半部 $[L, mid]$ 算出來的 DP 值,更新右半部 $[mid + 1, R]$ 的 DP 值。
      \item 遞迴計算右半部的 $[mid + 1, R]$ DP 值。
    \end{itemize}
    \item 複雜度為 $O(N \log^2 N)$
    \item 範例 code 麻煩參考講義~
  \end{itemize}
\end{frame}

\begin{frame}{\btitle{基於轉移點單調的唬爛}}
  \begin{itemize}
    \item 使用這個方法之前,要先決定一個魔法常數 $magic$,接著,該方法的執行過程如下:
    \item 假設現在在計算 $dp(i)$,並已知 $dp(i - 1)$ 的轉移點為 $p$
    \item 嘗試從 $[p, p + magic]$ 轉移到 $dp(i)$,如果發現一個比 $p$ 好的點,把 $p$ 更新成新的轉移點,並重新執行這部份
    \item 嘗試從 $[i - magic, i]$ 轉移到 $dp(i)$,如果發現一個比 $p$ 好的點,就更新 $p$
    \item 複雜度為 $O(magic \times N)$,\textbf{但是並沒有保證正確性!}
    \item 但是不失為一個\textbf{唬爛好方法}
  \end{itemize}
\end{frame}


\end{document}
